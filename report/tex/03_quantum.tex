\chapter{Quantum computing}

\section{The qubit}
The quantum bit, the qubit, is the building block of quantum computing.
Like the classical binary digit it can be either 0 or 1.
But being quantum, these are quantum states, $\ket{0}$ and $\ket{1}$, and the qubit can be in any superposition of these states.
The state of the qubit lies in a two-dimensional vector space, and the states $\ket{0}$ and $\ket{1}$ are basis vectors, known as the computational basis states.
Thus, the state of a qubit can be expressed as
\begin{equation}
    \ket{\psi} = \alpha \ket{0} + \beta \ket{1} = \begin{pmatrix} \alpha \\ \beta \end{pmatrix}
    \label{eq:qubit}
\end{equation}
where $\alpha$ and $\beta$ are any numbers, even complex ones.
The only requirement is that the state is normalised, i.e. $\vert\alpha\vert^2 + \vert\beta\vert^2 = 1$.
In particular, the qubit state lies in the Hilbert space $\mathcal{H} = \mathbb{C}^2$.


\subsection{The Bloch sphere}
A useful tool for visualising the state of a qubit is the Bloch sphere.
First, it should be noted for states on the form \cref{eq:qubit} are not unique, only the relative complex phase matters.
There is a global phase which is not measurable, and thus not relevant for the state of the qubit.
Therefore, taking also the normalisation requirement into account, the state of the qubit can be expressed as
\begin{equation}
    \ket{\psi} = \cos\left(\frac{\theta}{2}\right) \ket{0} + e^{i\phi} \sin\left(\frac{\theta}{2}\right) \ket{1}
    \label{eq:bloch}
\end{equation}
where $\theta, \phi \in \mathbb{R}$.
Interpreting $\theta$ as the polar angle and $\phi$ the azimuthal angle, the state of the qubit can be identified with a point a sphere, the Bloch sphere.
There, the state $\ket{0}$ is typically thought of as the north pole, and $\ket{1}$ as the south pole.
\Cref{fig:bloch} shows the Bloch sphere, and the state of the qubit in \cref{eq:bloch}.

% insert figure place holder
\begin{figure}
    \centering
    \includegraphics[width=0.5\textwidth]{example-image-a}
    \caption{The Bloch sphere figure placeholder.}
    \label{fig:bloch}
\end{figure}

\subsection{Multiple qubits}
Although the continuous nature of the qubit is indeed useful, the true power of quantum computers lie in how multiple qubits interact.
Having multiple qubits allows for the creation of entanglement, which is a key feature of quantum computing.
The state of multiple qubits can be expressed using the tensor product as
\begin{equation}
    \ket{\psi_1 \psi_2 \cdots  \psi_n} = \ket{\psi_1} \otimes \ket{\psi_2} \otimes \cdots \otimes \ket{\psi_n}.
    \label{eq:tensor}
\end{equation}
What makes this so powerful is that the state of a multi-qubit system does not have to be a product state. In can be anything on the form
\begin{equation}
    \ket{\psi_1 \psi_2 \cdots  \psi_n}
    = c_1 \ket{0\dots 00} + c_2 \ket{0\dots 01} + \cdots + c_{2^n} \ket{1\dots 11}
    = \begin{pmatrix}
        c_1 \\ c_2 \\ \vdots \\ c_{2^n}
    \end{pmatrix}
    \in \mathbb{C}^{2^n},
    \label{eq:superposition}
\end{equation}
which means that with $n$ qubits, the system can be in any superposition of the $2^n$ basis states.
Operating on several qubits then, one can do linear algebra in an exponentially large space.

\section{Operating with qubits}
\subsection{Single-qubit gates}
To do an operation on one or more qubits, a unitary matrix is applied to the state, where the unitarity is needed for states to remain normalised.
These operations are often thought of as gates, paralleling the classical gates in digital logic.
The most basic gates are the Pauli gates, which are the $X$, $Y$ and $Z$ gates:
\begin{align}
    X =\ket{0}\bra{1} + \ket{1}\bra{0} & = \begin{pmatrix} 0 & 1 \\ 1 & 0 \end{pmatrix},  \\
    Y =\ket{0}\bra{1} - \ket{1}\bra{0} & = \begin{pmatrix} 0 & -i \\ i & 0 \end{pmatrix}, \\
    Z =\ket{0}\bra{0} - \ket{1}\bra{1} & = \begin{pmatrix} 1 & 0 \\ 0 & -1 \end{pmatrix}.
\end{align}
These gates are half turns around the $x$, $y$ and $z$ axes of the Bloch sphere, respectively. The $X$ gate is also known as the NOT gate, as it mirrors the classical NOT gate by mapping $\ket{0}$ to $\ket{1}$ and vice versa.

The Hadamard gate
\begin{equation}
    H = \frac{1}{\sqrt{2}} \begin{pmatrix} 1 & 1 \\ 1 & -1 \end{pmatrix}
\end{equation}
is a rotation around the $x$-axis by $\pi/2$. It may be the most important gate in quantum computing, and is used to create superpositions of the computational basis states.

The $R_X$, $R_Y$ and $R_Z$ gates are rotations around the $x$, $y$ and $z$ axes, respectively, by an arbitrary angle $\theta$:
\begin{align*}
    R_X(\theta) & = \begin{pmatrix} \cos\left(\frac{\theta}{2}\right) & -i \sin\left(\frac{\theta}{2}\right) \\ -i \sin\left(\frac{\theta}{2}\right) & \cos\left(\frac{\theta}{2}\right) \end{pmatrix}, \\
    R_Y(\theta) & = \begin{pmatrix} \cos\left(\frac{\theta}{2}\right) & -\sin\left(\frac{\theta}{2}\right) \\ \sin\left(\frac{\theta}{2}\right) & \cos\left(\frac{\theta}{2}\right) \end{pmatrix},      \\
    R_Z(\theta) & = \begin{pmatrix} e^{-i\frac{\theta}{2}} & 0 \\ 0 & e^{i\frac{\theta}{2}} \end{pmatrix}.
\end{align*}
These parametrised gates will be useful in \cref{sec:vqa}.

\subsection{Multi-qubit gates}
The most important multi-qubit gate is the controlled-$X$ gate, also known as the CNOT, which is a controlled version of the $X$ gate.
Being controlled means that it only acts on the second qubit if the first qubit is in the state $\ket{1}$.
Of course, the first qubit may be in a superposition, and the CNOT this way allows for the creation of entanglement between the two qubits.
The CNOT gate is defined as
\begin{equation}
    \text{CNOT} = \begin{pmatrix} 1 & 0 & 0 & 0 \\ 0 & 1 & 0 & 0 \\ 0 & 0 & 0 & 1 \\ 0 & 0 & 1 & 0 \end{pmatrix}.
\end{equation}

\subsection{Quantum circuits}
The operations on qubits are often described using quantum circuits, which are a graphical representation of the operations on the qubits, the quantum algorithms.
They are read from left to right.
It is standard procedure to assume all qubits start in the state $\ket{0}$.
For instance, using an $H$-gate to create a superposition before applying a CNOT gate, can be expressed as
\begin{equation}
    \begin{quantikz}
        \lstick{$\ket{0}$} & \gate{H} & \ctrl{1} & \qw \\
        \lstick{$\ket{0}$} & \qw & \targ{} & \qw
    \end{quantikz}
\end{equation}

\section{Limitations of NISQ hardware}
Quantum hardware have been physically realised and even outperforms classical computers in very contrived situations, but the hardware is still very limited.
The hardware is limited in the number of qubits, the connectivity between the qubits, and the noise and decoherence of the qubits.
It is believed that quantum hardware will continue to improve and eventually perform demanding algorithms like Shor's for large numbers.
Still, the era dubbed NISQ (Noisy Intermediate-Scale Quantum) is the first step, and to make use of the hardware, algorithms must take these limitations into consideration.

Noise and decoherence severely limits how large circuits can be run on the hardware.
Decoherence refers to the fact that the qubits are not isolated from the environment, and may be ruined by the environment, e.g. electrical noise from the control electronics.
Furthermore, with the continuous nature of quantum states, minor errors can compound.
If for instance a qubit is to be rotated many times, a small error in the rotation may cause the qubit to be rotated by a large angle.
Because of this, NISQ algorithms must be shallow, meaning that the amount of gates applied before measurement is small.

Another limiting factor is the amount of qubits.
Current hardware has around 10-100 qubits, which though still may be enough to express states too large to be expressed on classical computers, is not enough to perform the most demanding algorithms.
With more qubits, error correction could be used to mitigate the effects of noise and decoherence, but this would require many more qubits than are currently available.
Another current limitation is the connectivity between the qubits.
Not all qubits are directly linked, which means that applying a multi-qubit gate may require intermediate swapping of qubits.
This increases circuit depth which in turn increases the error rate.

