\section{Limitations of NISQ hardware}
Quantum hardware have been physically realised and even outperforms classical computers in very contrived situations, but the hardware is still very limited.
The hardware is limited in the number of qubits, the connectivity between the qubits, and the noise and decoherence of the qubits.
It is believed that quantum hardware will continue to improve and eventually perform demanding algorithms like Shor's for large numbers.
Still, the era dubbed NISQ (Noisy Intermediate-Scale Quantum) is the first step, and to make use of the hardware, algorithms must take these limitations into consideration.

Noise and decoherence severely limits how large circuits can be run on the hardware.
Decoherence refers to the fact that the qubits are not isolated from the environment, and may be ruined by the environment, e.g., electrical noise from the control electronics.
Furthermore, with the continuous nature of quantum states, minor errors can compound.
If for instance a qubit is to be rotated many times, a small error in the rotation may cause the qubit to be rotated by a large angle.
Because of this, NISQ algorithms must be shallow, meaning that the amount of gates applied before measurement is small.

Another limiting factor is the amount of qubits.
Current hardware has around 10 to 100 qubits, which though still may be enough to express states too large to be expressed on classical computers, is not enough to perform the most demanding algorithms.
With more qubits, error correction could be used to mitigate the effects of noise and decoherence, but this would require many more qubits than are currently available; recent estimates require millions of noisy qubits would be needed to break RSA encryption \cite{gidney2021}.
Another current limitation is the connectivity between the qubits.
Not all qubits are directly linked, which means that applying a multi-qubit gate may require intermediate swapping of qubits.
This increases circuit depth which in turn increases the error rate.