\section{Quantum operations}
\label{sec:quantum_operations}
\subsection{Single-qubit gates}
To operate on one or more qubits, a unitary operation is applied to the state.
This is a computational interpretation of the unitary time evolution resulting from a Hamiltonian acting on the (closed) quantum system, described by the second postulate of quantum mechanics and the Schrödinger equation.
As the operations are unitary, a pure state remains pure.
These operations are often thought of as gates, paralleling the classical gates in digital logic.
Mathematically, with a finite number of qubits, a unitary gate $U$ can be expressed as matrices acting on the state vector, $\ket{\psi}$, as
\begin{equation}
    \ket{\psi'} = U\ket{\psi},
\end{equation}
where $\ket{\psi'}$ is the resulting state.

The most basic gates are the Pauli gates, which are applications of the Pauli matrices from \cref{eq:pauli} and are as gates simply denoted as $X$, $Y$, and $Z$.
These gates can be seen as half turns around the $x$-, $y$- and $z$-axes, respectively, of the Bloch sphere.
The $X$-gate is also known as the NOT gate, as it mirrors the classical NOT gate by mapping $\ket{0}$ to $\ket{1}$ and vice versa.
It is however more general, being also applicable to superposition states.

The Hadamard gate,
\begin{equation}
    H = \frac{1}{\sqrt{2}} \begin{pmatrix} 1 & 1 \\ 1 & -1 \end{pmatrix},
\end{equation}
is a rotation around the line between the $x$- and $z$-axes by $\pi/2$.
It is an important gate in quantum computing, as it is used to create superpositions of the computational basis states.
Applied on the initial $\ket{0}$ state, it creates the entangled state $\frac{1}{\sqrt{2}}(\ket{0} + \ket{1})$.
Two consecutive applications thereof returns the state to the initial state, as can be seen from the matrix squaring to the identity.

The $R_X$-, $R_Y$- and $R_Z$-gates are rotations around the $x$-, $y$- and $z$-axes, respectively, by an arbitrary angle $\theta$:
\begin{align*}
    R_X(\theta) & = \begin{pmatrix} \cos\left(\frac{\theta}{2}\right) & -i \sin\left(\frac{\theta}{2}\right) \\ -i \sin\left(\frac{\theta}{2}\right) & \cos\left(\frac{\theta}{2}\right) \end{pmatrix}, \\
    R_Y(\theta) & = \begin{pmatrix} \cos\left(\frac{\theta}{2}\right) & -\sin\left(\frac{\theta}{2}\right) \\ \sin\left(\frac{\theta}{2}\right) & \cos\left(\frac{\theta}{2}\right) \end{pmatrix},      \\
    R_Z(\theta) & = \begin{pmatrix} e^{-i\frac{\theta}{2}} & 0 \\ 0 & e^{i\frac{\theta}{2}} \end{pmatrix}.
\end{align*}
These parametrised gates will be useful in \cref{sec:vqa}.

\subsection{Multi-qubit gates}
Multi-qubit gates are gates that act non-trivially on more than one qubit.
The most used multi-qubit gate is the controlled $X$-gate, also known as the CNOT.
Being controlled means that it only acts on the second qubit if the first qubit is in the state $\ket{1}$.
Of course, the first qubit may be in a superposition, and the CNOT this way allows for the creation of entanglement between the two qubits.
If the first qubit has probability amplitude $\alpha$ of being in the state $\ket{1}$, the second qubit will have probability amplitude $\alpha$ of being flipped.
The CNOT-gate, acting on the leftmost qubit in the tensored two-qubit system can be expressed in matrix form as
\begin{equation}
    \text{CNOT} = \begin{pmatrix} 1 & 0 & 0 & 0 \\ 0 & 1 & 0 & 0 \\ 0 & 0 & 0 & 1 \\ 0 & 0 & 1 & 0 \end{pmatrix}.
\end{equation}

In theory, any unitary single-qubit operation can be controlled.
However, it is often only the CNOT that is used is implemented in the hardware.
Another interesting two-qubit gate is the controlled $Z$-gate, CZ, expressible as the matrix
\begin{equation}
    \text{CZ} = \begin{pmatrix} 1 & 0 & 0 & 0 \\ 0 & 1 & 0 & 0 \\ 0 & 0 & 1 & 0 \\ 0 & 0 & 0 & -1 \end{pmatrix}.
\end{equation}
Because it only alters the amplitude of $\ket{11}$, it does not actually matter which qubit is the control and which is the target.

\subsection{Observables and measurements}
For an output to be obtained from a quantum computer, a measurement must be performed.
This is typically done at the end of all operations and of all qubits, thus yielding a single output of zeros and ones, a bit-string.

As described by the third postulate of quantum mechanics, any observable quantity has a corresponding Hermitian operator $A$, spectrally decomposable as $A = \sum_{i} \lambda_i P_i$, where $\lambda_i$ are the (necessarily real) eigenvalues and $P_i$ are the corresponding projectors onto the eigenspaces.
When measuring, the probability of obtaining the outcome $\lambda_i$ is given by
\begin{equation}
    p_i = \bra{\psi} P_i \ket{\psi},
\end{equation}
where $\ket{\psi}$ is the state before the measurement.
The state after the measurement is then given by
\begin{equation}
    \ket{\psi} \mapsto \frac{P_i \ket{\psi}}{\sqrt{\bra{\psi} P_i \ket{\psi}}},
\end{equation}
where $P_i$ is randomly chosen according to the probabilities $p_i$.
It is one of Nature's great mysteries what exactly a measurement is and even more so how and why it is different from the unitary evolution described by the second postulate.
In the quantum computational setting, it can be thought of as taking a random sample with the probabilities as given by the above equation.

In this probabilistic settings, it is often the underlying probabilities are what is of interest.
Therefore, many measurements will be performed.
Usually, these results are averaged to obtain an estimate, but more complicated post-processing methods are also possible.
For instance, neural networks have shown useful properties in regard of reducing variance in the estimates, though at the cost of some bias \cite{torlai2020}.

Canonically, the computational $Z$-basis is used for measurements, and it is usually the only basis for which measurements are physically implemented in a quantum computer.
When measuring in the computational basis in which a state is expressed, as \cref{eq:superposition}, the probabilities are simply given by the absolute square of the coefficients.
To virtually measure another observable, a change of basis is performed.
This is achieved by applying a unitary transformation before measurement.

Measurements may be done in the middle of a computation and be used to control gates.
If the qubits are entangled, measuring one will affect the measurement probabilities of others.
Using such intermediate measurements is a way of introducing non-linearities in the otherwise unitary nature of the quantum world.

\subsection{Quantum circuits}
The operations on qubits are often described using quantum circuits, which are a graphical representation of the operations on the qubits, the quantum algorithms.
They are read from left to right.
It is standard procedure to assume all qubits start in the state $\ket{0}$.
Gates are generally written as boxes with the name of the gate inside.

A simple example is the circuit
\begin{equation}
    \begin{quantikz}
        \lstick{$\ket{0}$} & \gate{H} & \qw \\
        \lstick{$\ket{0}$} & \gate{H} & \qw
    \end{quantikz},
\end{equation}
which prepares the state
$
    \frac{1}{2}(\ket{00} + \ket{01} + \ket{10} + \ket{11})
    =
    \frac{1}{\sqrt{2}}
    \left(
    \ket{0} + \ket{1}
    \right)
    \otimes
    \frac{1}{\sqrt{2}}
    \left(
    \ket{0} + \ket{1}
    \right)
$.
This is a pure state with no entanglement, and so the measurement probabilities of the two qubits are independent.

Slightly more interesting is the circuit
\begin{equation}
    \begin{quantikz}
        \lstick{$\ket{0}$} & \gate{H} & \ctrl{1} & \qw \\
        \lstick{$\ket{0}$} & \qw & \targ{} & \qw
    \end{quantikz}
\end{equation}
in which the first qubit is put into a superposition using the Hadamard gate before a CNOT gate is applied to the second, controlled on the first.
This creates the state $\frac{1}{\sqrt{2}}\left(\ket{00} + \ket{11}\right)$.
The measurement probabilities of the two qubits are now correlated; if the first qubit is measured to be $\ket{1}$, the second will always be $\ket{1}$ and vice versa.
The probability of measuring the qubits to be different is nil.

To create a mixed state, an intermediate measurement can be used to control a gate.
For instance, the circuit
\begin{equation}
    \begin{quantikz}
        \lstick{$\ket{0}$} & \gate{H} & \meter{} & \cwbend{1} \\
        \lstick{$\ket{0}$} & \qw & \qw & \gate{X} & \qw
    \end{quantikz}
\end{equation}
places the second qubit in the mixed state $\frac{1}{2}(\ketbra{0}{0} + \ketbra{1}{1})$.
If it were immediately to be measured, it would have a 50\% chance of being $\ket{0}$ and a 50\% chance of being $\ket{1}$.
The uncertainty is only classical, and it could therefore not be used to create entanglement or for any other quantum spookiness.

\subsection{Quantum supremacy}
Exponential speed-ups do not come for free.
Quantum computers do only solve certain problems more efficiently than classical computers, and finding the algorithms to do so is not trivial.
Shor's algorithm has time complexity $O((\log N)^3)$ while the most efficient known classical algorithm, the general number field sieve, is sub-exponential with a time complexity on the form $\Omega(k^{\frac{1}{3}}\log^{2/3}k)$, where $k=O(2^N)$ \cite{dervovic2018}.
To solve linear system, there is the HHL algorithm with time complexity $O(\log(N)\kappa^2)$, where $\kappa$ is the condition number.
This is an exponential speed-up over the fastest known classical algorithm\footnotemark{}, which has time complexity $O(N \kappa)$.
Still, these are non-trivial algorithms, not yet usable in practice and that were not easily found.

\footnotetext{
    Given that the condition number does not grow exponentially.
    There are also difficulties in loading the data into the quantum computer and extracting the solution that could negate any exponential speed-up.
    For details, see \cite{aaronson2015}.
}

Polynomial speed-ups are perhaps more easily found.
For example, the Grover algorithm which is used to search for an element in an unsorted list has time complexity $O(\sqrt{N})$ \cite{grover1996}.
Classically, this can not be done in less than $O(N)$ time.
It can be proven that the Grover algorithm is optimal \cite{zalka1999}, so for this problem, an exponential speed-up is impossible.
This algorithm and the more general amplitude amplification on which it builds solves very general problems and are often used subroutines to achieve quadratic speed-ups in other algorithms.
Being only a quadratic speed-up, it is not as impressive as the exponential speed-ups, and achieving quantum supremacy in that manner would require larger quantum computers than if the speed-up were exponential.

It is proven that the class of problems quantum computers can solve in polynomial time (with high probability), BQP, contains the complexity class P \cite{nielsen2012}.
This follows from the fact than quantum computers can do any classical algorithm.
Since quantum computers can solve problems like integer factorisation and discrete logarithms efficiently, it is believed that BQP is strictly greater than P, but as whether P = NP remains unknown, these problems could actually be in P.
In a similar vein, NP-complete problems are believed to lie outside BQP.