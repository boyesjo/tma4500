\chapter{Introduction}

\section{Other work}

\subsection{Quantum optimization using variational algorithms on near-term quantum devices}
While quantum computers are constantly improving, those of the near future will be limited by the number of qubits available, their connectivity and the amount of noise and error in the operations and measurements. Errors occur both due to quantum decoherence, being a significant when many operations are done and because of inaccuracies in the operations done themselves. Limited connectivity means that not all qubits are directly connected, thus requiring several intermediate operations such as swapping in order for qubits far apart to interact. Subsequently, the procedures that can be done are limited, due to the aforementioned errors only being reinforced by  this overhead.

Current quantum computers vary in several ways: the total amount of qubits, their connectivity, how many gates can be applied before errors or decoherence ruins the results, which gates are physically implemented in the hardware and the degree of gate parallelism possible. In order to make quantum computers comparable, Moll et al. define in \cite{moll2018} a metric they denote \textit{quantum volume}, which effectively describes a quantum computer's ability to perform useful quantum algorithms.