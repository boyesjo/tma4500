\chapter{Machine learning}
\label{chap:ml}

Machine learning lies at the intersection of statistics, computer science and optimisation.
The central idea is to design an algorithm that uses data to solve a problem, and in so avoid explicitly programming a solution.
With ever more data available and with ever more powerful computers, machine learning has become a powerful tool in many fields, solving problems previously thought intractable.
Such algorithms or models can be used for a plethora of tasks, which are mainly divided into three main categories:
\begin{itemize}

      \item
            \textit{Supervised learning:} Given data with corresponding labels, find the relationship and try to assign correct labels to new, unseen data.

      \item
            \textit{Unsupervised learning:} Given data, find some underlying structure, patterns, properties or relationships, such as clusters or outliers.

      \item
            \textit{Reinforcement learning:} Given an environment with a set of possible actions, such as a game, explore different strategies and determine one that optimises some reward.

\end{itemize}
Only the first thereof will be explicitly considered in this thesis, though much of the theory and results can be extended to the latter two.
This chapter is hardly a comprehensive introduction to machine learning, but rather a brief overview of the most important concepts and techniques, serving as a reference point for the later endeavours into quantum machine learning.
Some familiarity with basic probability theory is assumed.

% \subimport{}{intro}
\subimport{}{new}
\subimport{}{nn}