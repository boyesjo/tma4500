\section{QCNN with mid-circuit measurements}
\label{sec:qcnn2}
A more complex QCNN structure was described by \textcite{pesah2021}, where the pooling modules measure a qubit and use the result to control a unitary gate on its neighbour.
The use of mid-circuit measurements complicates the circuit and its implementation, but allows for a non-linear and potentially more powerful model.
With the aim of seeing if this provides any benefit, such a QCNN was implemented and compared with that of \cref{sec:qcnn1} on the data thence.

\subsection{Model}
First, the data was encoded using angle encoding in the $X$-direction.
The convolutional layers consisted of pairwise $W$-gates, a mix of parametrised rotations and CNOTs, with total of 15 parameters
As circuits, they appear as
\begin{equation}
    \begin{quantikz}[column sep=0.5em]
        \qw
        &
        \gate[wires=2]{W}
        &
        \qw
        \midstick[2,brackets=none]{=}
        &
        \gate{R_G}
        &
        \targ{}
        &
        \gate{R_Z}
        &
        \ctrl{1}
        &
        \qw
        &
        \targ{}
        &
        \gate{R_G}
        &
        \qw
        \\
        \qw
        &
        &
        \qw
        &
        \gate{R_G}
        &
        \ctrl{-1}
        &
        \gate{R_Y}
        &
        \targ{}
        &
        \gate{R_Y}
        &
        \ctrl{-1}
        &
        \gate{R_G}
        &
        \qw
    \end{quantikz},
    \label{eq:w_gate}
\end{equation}
where $R_G$ are general rotations around all three axes and so have three parameters.

The pooling layers consisted of a measurement and a conditional single-qubit unitary gate.
Without any particular recommendation in the original paper, a simple $X$-gate was used.
Combined, the convolution and pooling modules appear as in \cref{fig:qcnnm}.

Like in \cref{sec:qcnn}, the network used 8 qubits to handle the 8-dimensional data.
Three convolutional and pooling layers were used, reducing the data from 8 to 2 dimensions
Lastly, a final general two-qubit ansatz was used, and a single qubit was measured and interpreted as a prediction.
This was a simple parametrised entangler, similar to the \texttt{RealAmplitudes} in \cref{sec:qnn-vs-nn}, but with $X$-rotations.
In total, the network had 154 parameters.

\begin{figure}
    \centering
    \begin{quantikz}
        % \lstick[wires=4]{$\ket{\psi(\bm{x})}$}
        &
        \qw
        \gategroup[
            wires=4,
            steps=2,
            style={dashed, rounded corners, inner sep=2pt},
            label style={label position=below, anchor=north, yshift=-0.2cm},
        ]{Convolution}
        &
        \gate[wires=2]{W}
        &
        \qw
        &
        \meter{}
        \gategroup[
            wires=4,
            steps=2,
            style={dashed, rounded corners, inner sep=2pt},
            label style={label position=below, anchor=north, yshift=-0.2cm},
        ]{Pooling}
        &
        \cwbend{1}
        \\
        &
        \gate[wires=2]{W}
        &
        \qw
        &
        \qw
        &
        \qw
        &
        \gate{X}
        &
        \qw
        \\
        &
        \qw
        &
        \gate[wires=2]{W}
        &
        \qw
        &
        \qw
        &
        \gate{X}
        &
        \qw
        \\
        &
        \qw
        &
        \qw
        &
        \qw
        &
        \meter{}
        &
        \cwbend{-1}
    \end{quantikz}
    \caption{
        QCNN convolution and pooling layer structure with intermediate measurements.
        Some encoded data or already pooled data enter the convolution layer where parametrised gates $W$ entangle the qubits.
        The pooling modules measure a qubit and use the result to control a unitary gate on a neighbour (here the $X$-gate).
    }
    \label{fig:qcnnm}
\end{figure}

\subsection{Implementation}
The QCNN was implemented using the PennyLane framework and trained with the Adam optimiser with a learning rate of 0.01.
With the PennyLane implementation, there were no problems using gradient based optimisation.
This prompted the reimplementation of the former QCNN, which was here also trained with the same Adam optimiser.

\subsection{Results}
The results are shown in \cref{fig:qcnnm_training}.
The training loss was lower than for the QCNN in \cref{sec:qcnn}, despite the fewer iterations, showing expected advantages of using gradient based optimisation.
Furthermore, the new model with intermediate measurements achieves a perfect accuracy on both the training and test sets in only 10 iterations.
Looking at the loss curves, the model with intermediate measurements seems only to converge slightly faster than the model without intermediate measurements.

However, by extending the model without mid-circuit measurements to have a more comparable parameter count of 231, it overtakes the model with intermediate measurements in terms of training loss.
This extension was done by switching out the convolutional gates from \cref{eq:qcnn_conv_gate} with the $W$-gates from \cref{eq:w_gate}.

\begin{figure}
    \centering
    \begin{subfigure}{0.49\textwidth}
        \centering
        \begin{tikzpicture}
            \begin{axis}[
                    width=\textwidth,
                    height=\textwidth,
                    xlabel={Iteration},
                    ylabel={MSE},
                    % legend pos=north west,
                    % legend style={at={(0.5,1.03)},anchor=north},
                    grid=major,
                    legend pos=north east,
                    ymode=log,
                ]
                \addplot[mark=none, color=blue] table[x=Iteration, y=Loss, col sep=comma] {../code/qcnn_pennylane/results_intermediate.csv};
                \addplot[mark=none, color=red] table[x=Iteration, y=Loss, col sep=comma] {../code/qcnn_pennylane/results_simple.csv};
                \addplot[mark=none, color=green] table[x=Iteration, y=Loss, col sep=comma] {../code/qcnn_pennylane/results_simple_ext.csv};
                \legend{
                    With intermediate,
                    Without,
                    Without extended
                }
            \end{axis}
        \end{tikzpicture}
        \caption{}
        \label{fig:qcnnm_loss}
    \end{subfigure}
    \begin{subfigure}{0.49\textwidth}
        \centering
        \begin{tikzpicture}
            \begin{axis}[
                    width=\textwidth,
                    height=\textwidth,
                    xlabel={Iteration},
                    ylabel={Accuracy},
                    % legend pos=north west,
                    % legend style={at={(0.5,1.03)},anchor=north},
                    grid=major,
                    legend pos=south east,
                    xmax=35,
                ]
                \addplot[mark=none, color=blue] table[x=Iteration, y=Accuracy, col sep=comma] {../code/qcnn_pennylane/results_intermediate.csv};
                \addplot[mark=none, color=blue, dashed] table[x=Iteration, y=Test accuracy, col sep=comma] {../code/qcnn_pennylane/results_intermediate.csv};
                \addplot[mark=none, color=red] table[x=Iteration, y=Accuracy, col sep=comma] {../code/qcnn_pennylane/results_simple.csv};
                \addplot[mark=none, color=red, dashed] table[x=Iteration, y=Test accuracy, col sep=comma] {../code/qcnn_pennylane/results_simple.csv};
                \addplot[mark=none, color=green] table[x=Iteration, y=Accuracy, col sep=comma] {../code/qcnn_pennylane/results_simple_ext.csv};
                \addplot[mark=none, color=green, dashed] table[x=Iteration, y=Test accuracy, col sep=comma] {../code/qcnn_pennylane/results_simple_ext.csv};
                \legend{
                    With intermediate,
                    , %With intermediate (test),
                    Without,
                    , %Without (test),
                    Without extended,
                    , %Without extended (test)
                }
            \end{axis}
        \end{tikzpicture}
        \caption{}
        \label{fig:qcnnm_acc}
    \end{subfigure}
    \caption{
        Training of QCNNs with and without intermediate measurements.
        With intermediate refers to the QCNN with intermediate measurements of \cite{pesah2021}.
        Without refers to the QCNN without intermediate measurements from \Cref{sec:qcnn}, while without extended is the same with a more expressive convolutional layer.
        (a) loss (mean square error) during training.
        (b) accuracy on the training and test sets.
        Solid lines are for the training set, dashed lines for the test set.
        Note the reduced $x$-axis range; all models achieve perfect accuracy quickly.
    }
    \label{fig:qcnnm_training}
\end{figure}