\chapter{Simulated comparisons of QNNs}
\label{chap:comparison}

To see if quantum machine learning indeed is a viable alternative to classical machine learning, some empirical comparisons are in order.
In this chapter, three quantum neural networks are implemented and tested.
First, the convergence of a quantum neural network is compared to a classical neural network.
Second, a quantum convolutional neural network is implemented and tested.
Lastly, the convolutional network is expanded to include mid-circuit measurements.

All simulations were performed on an Apple M1 Pro processor, with the Qiskit \cite{qiskit} and PennyLane \cite{pennylane} frameworks for quantum computing and PyTorch \cite{pytorch} for classical machine learning and optimisation.
Simulating quantum systems is exponentially hard\footnote{After all, it prompted the conception of quantum computing in the first place, q.v. the introduction (\cref{chap:intro}).}, so the number of qubits is limited.
Only systems of up to around a dozen qubits are easily simulated on consumer-grade hardware.
To what extent the results hence applies to larger systems is hard to say.

\subimport{}{qnn_vs_nn}
\subimport{}{qcnn}
\subimport{}{qcnn2}