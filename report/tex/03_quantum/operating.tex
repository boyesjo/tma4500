\section{Operating with qubits}
\subsection{Single-qubit gates}
To operate on one or more qubits, a unitary operation is applied to the state, where the unitarity is needed for states to remain normalised.
With the finite number of qubits, these operations can be expressed as matrices.
These operations are often thought of as gates, paralleling the classical gates in digital logic.
The most basic gates are the Pauli gates, which are the $X$, $Y$ and $Z$ gates:
\begin{align}
    X =\ket{0}\bra{1} + \ket{1}\bra{0} & = \begin{pmatrix} 0 & 1 \\ 1 & 0 \end{pmatrix},  \\
    Y =\ket{0}\bra{1} - \ket{1}\bra{0} & = \begin{pmatrix} 0 & -i \\ i & 0 \end{pmatrix}, \\
    Z =\ket{0}\bra{0} - \ket{1}\bra{1} & = \begin{pmatrix} 1 & 0 \\ 0 & -1 \end{pmatrix}.
\end{align}
These gates can be seen as half turns around the $x$, $y$ and $z$ axes, respectively, of the Bloch sphere.
The $X$ gate is also known as the NOT gate, as it mirrors the classical NOT gate by mapping $\ket{0}$ to $\ket{1}$ and vice versa, though it of course is more general, being applicable to superposition states too.

The Hadamard gate
\begin{equation}
    H = \frac{1}{\sqrt{2}} \begin{pmatrix} 1 & 1 \\ 1 & -1 \end{pmatrix}
\end{equation}
is a rotation around the $x$-axis by $\pi/2$.
It is a very important gate in quantum computing, as it is used to create superpositions of the computational basis states.


The $R_X$, $R_Y$ and $R_Z$ gates are rotations around the $x$, $y$ and $z$ axes, respectively, by an arbitrary angle $\theta$:
\begin{align*}
    R_X(\theta) & = \begin{pmatrix} \cos\left(\frac{\theta}{2}\right) & -i \sin\left(\frac{\theta}{2}\right) \\ -i \sin\left(\frac{\theta}{2}\right) & \cos\left(\frac{\theta}{2}\right) \end{pmatrix}, \\
    R_Y(\theta) & = \begin{pmatrix} \cos\left(\frac{\theta}{2}\right) & -\sin\left(\frac{\theta}{2}\right) \\ \sin\left(\frac{\theta}{2}\right) & \cos\left(\frac{\theta}{2}\right) \end{pmatrix},      \\
    R_Z(\theta) & = \begin{pmatrix} e^{-i\frac{\theta}{2}} & 0 \\ 0 & e^{i\frac{\theta}{2}} \end{pmatrix}.
\end{align*}
These parametrised gates will be useful in \cref{sec:vqa}.

\subsection{Multi-qubit gates}
The most important multi-qubit gate is the controlled-$X$ gate, also known as the CNOT, which is a controlled version of the $X$ gate.
Being controlled means that it only acts on the second qubit if the first qubit is in the state $\ket{1}$.
Of course, the first qubit may be in a superposition, and the CNOT this way allows for the creation of entanglement between the two qubits.
If the first qubit has probability amplitude $\alpha$ of being in the state $\ket{1}$, the second qubit will have probability amplitude $\alpha$ of being flipped.
The CNOT gate can be expressed in matrix form as
\begin{equation}
    \text{CNOT} = \begin{pmatrix} 1 & 0 & 0 & 0 \\ 0 & 1 & 0 & 0 \\ 0 & 0 & 0 & 1 \\ 0 & 0 & 1 & 0 \end{pmatrix}.
\end{equation}

\subsection{Observables and measurements}
For an output to be obtained from a quantum computer, a measurement must be performed.
This is typically done at the end of all operations and of all qubits, thus yielding a single output of zeroes and ones, a bit-string.
It is important to note that the measurement is not a deterministic process, but rather a probabilistic one.
Often, the underlying probabilities are what is of interest.
Therefore, many measurements are performed.
Usually, these results are averaged to obtain an estimate, but more complicated post-processing methods are also possible.
For instance, neural networks have shown useful properties in regard of reducing variance in the estimates, though at the cost of some bias \cite{torlai2020}.

The $Z$ basis is the canonical basis for measurements, but any other basis can be used, at least in theory.
Often, only canonical basis measurements are implemented in the hardware.
Using another basis can be done by properly preparing the state before the measurement.

Measurements may be done in the middle of operations and be used to control the operations.
If the qubits are entangled, measuring one can affect the measurement probabilities of others.
Using such intermediate measurements is a way of introducing non-linearities in the otherwise unitary nature of the quantum world.


\subsection{Quantum circuits}
The operations on qubits are often described using quantum circuits, which are a graphical representation of the operations on the qubits, the quantum algorithms.
They are read from left to right.
It is standard procedure to assume all qubits start in the state $\ket{0}$.
Gates are generally written as boxes with the name of the gate insides.
An example is the circuit
\begin{equation}
    \begin{quantikz}
        \lstick{$\ket{0}$} & \gate{H} & \ctrl{1} & \qw \\
        \lstick{$\ket{0}$} & \qw & \targ{} & \qw
    \end{quantikz}
\end{equation}
in which the first qubit is put into a superposition using the Hadamard gate before a CNOT gate is applied to the second, controlled on the first.
This creates the state $\frac{1}{\sqrt{2}}\left(\ket{00} + \ket{11}\right)$.