\chapter{Quantum machine learning}
\label{chap:qml}
How to combine quantum computing and machine learning is not easily answered.
In the discussion of quantum machine learning, it is standard practice to reference the four quadrants of \cref{tab:qml_quadrants}, first described by \textcite{schuld2018}.

\begin{table}
    \centering
    % increase padding
    \renewcommand{\arraystretch}{1.5}
    \caption{
        The four fundamental ways in which quantum computing and machine learning can be combined.
        CC: classical computer and classical data.
        CQ: classical computer and quantum data.
        QC: quantum computer and classical data.
        QQ: quantum computer and quantum data.
        From \cite{schuld2018}.
    }
    \begin{tabular}{cc|cc}
        \begin{tabular}{cc|cc}
                                           &                    & \multicolumn{2}{c}{\textbf{Computing device}}                    \\
                                           &                    & \multicolumn{1}{c|}{\textit{Classical}}       & \textit{Quantum} \\ \hline
            \multirow{2}{*}{\textbf{Data}} & \textit{Classical} & \multicolumn{1}{c|}{CC}                       & CQ               \\ \cline{2-4}
                                           & \textit{Quantum}   & \multicolumn{1}{c|}{QC}                       & QQ
        \end{tabular}
    \end{tabular}
    \label{tab:qml_quadrants}
\end{table}


Classical data being processed on classical computers is classical machine learning.
Though not explicitly linked to quantum computing, there are some ways in which quantum computing influences classical machine learning, such as the quantum-inspired application of tensor networks in \cite{felser2021}.

Using classical machine learning for quantum data includes improving quantum computers' general performance.
For example, with machine learning algorithms, the variance of the measurements can be reduced, as shown in \cite{torlai2020}.
Alternatively, advanced machine learning models like neural networks can be employed to describe quantum states more efficiently.

How to use quantum algorithms to solve machine learning problems is the main topic of this thesis and is what will be meant when quantum machine learning (QML) is mentioned.
QML concerns itself with how better to do what classical machine learning already does.
Quantum algorithms are most often advertised with speed-ups contra classical algorithms, often exponentially so as with Shor's algorithm.
While this is true, there are major difficulties in achieving these speed-ups.
However, there may be other advantages to be had, in terms of the amount of data needed to how much training has to be done.

The last quadrant of quantum computing handling quantum data includes quantum machine learning from for example quantum experiments or machine learning when the data is inherently quantum states.
With NISQ hardware, fully quantum procedures are difficult, so this field is not of immediate interest.
There is obviously much overlap with CQ as the data is quantum once encoded into the quantum computer, but as will be made clear, the encoding is such a big part of CQ that results thence are not necessarily applicable QQ.


\subimport{}{encoding}
\subimport{}{qnn}
% \subimport{}{comparisons}