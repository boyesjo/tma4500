\chapter{Quantum machine learning}
\label{chap:qml}
There are four fundamental ways to combine machine learning and quantum computing\cite{schuld2018}. One can differentiate between the type of the data being processed and the way of processing these data, see Table~\ref{tab:qml_quadrants}.
\begin{table}
    \centering
    % increase padding
    \renewcommand{\arraystretch}{1.5}
    \caption{
        The four fundamental ways to combine quantum computing and machine learning\cite{schuld2018}.
        % CC: classical computer and classical data.
        % CQ: classical computer and quantum data.
        % QC: quantum computer and classical data.
        % QQ: quantum computer and quantum data.
        % Adapted from \cite{schuld2018}.
    }
    \begin{tabular}{cc|cc}
        \begin{tabular}{cc|cc}
                                           &                    & \multicolumn{2}{c}{\textbf{Computing device}}                    \\
                                           &                    & \multicolumn{1}{c|}{\textit{Classical}}       & \textit{Quantum} \\ \hline
            \multirow{2}{*}{\textbf{Data}} & \textit{Classical} & \multicolumn{1}{c|}{CC}                       & CQ               \\ \cline{2-4}
                                           & \textit{Quantum}   & \multicolumn{1}{c|}{QC}                       & QQ
        \end{tabular}
    \end{tabular}
    \label{tab:qml_quadrants}
\end{table}



\textbf{CQ}. \textit{Classical data} being processed on \textit{classical computers} is classical machine learning.
Though not explicitly linked to quantum computing, there are some ways in which quantum computing influences classical machine learning, such as the quantum-inspired application of tensor networks \cite{felser2021}.

\textbf{QC}. Using \textit{classical machine learning} for \textit{quantum data} includes improving quantum computers' general performance.
For example, with machine learning algorithms, the variance of the measurements can be reduced \cite{torlai2020}.
Alternatively, advanced machine learning models like neural networks can be employed to describe quantum states more efficiently.

\textbf{QQ}. Using \textit{quantum computing} for handling \textit{quantum data} is another way to combine quantum computing and machine learning.
An example is to perform quantum machine learning directly on data from quantum experiments or machine learning when the data is inherently quantum states.
With NISQ hardware, fully quantum procedures are difficult, so this field is not of immediate interest.
There is obviously a large overlap with CQ as the data is quantum once encoded into the quantum computer, but as will be made clear, the encoding is such a big part of CQ that results thence are not necessarily applicable to QQ.

\textbf{CQ}. How to use \textit{quantum computers} to process \textit{classical data} is the main topic of this thesis and is what will be meant when quantum machine learning (QML) is mentioned.
QML concerns itself with how to improve classical machine learning in some way.
Quantum algorithms are most often advertised with speed-ups contra classical algorithms, often exponentially so as with Shor's algorithm.
While this is true, there are major difficulties in achieving these speed-ups.
However, there may be other advantages to be had, in terms of the amount of data needed to how much training has to be done.


\subimport{}{encoding}
\subimport{}{qnn}
% \subimport{}{comparisons}